%%%%%%%%%%%%%%%%%%%%%%%%%%%%%%%%%%%%%%%%%%%%%%%%%%%%%%%%%%%%%%%%%%%%%%
% LaTeX Template: Project Titlepage
%
% Source: http://www.howtotex.com
% Date: April 2011
% 
% This is a title page template which be used for articles & reports.
% 
% Feel free to distribute this example, but please keep the referral
% to howtotex.com
% 
%%%%%%%%%%%%%%%%%%%%%%%%%%%%%%%%%%%%%%%%%%%%%%%%%%%%%%%%%%%%%%%%%%%%%%
% How to use writeLaTeX: 
%
% You edit the source code here on the left, and the preview on the
% right shows you the result within a few seconds.
%
% Bookmark this page and share the URL with your co-authors. They can
% edit at the same time!
%
% You can upload figures, bibliographies, custom classes and
% styles using the files menu.
%
% If you're new to LaTeX, the wikibook is a great place to start:
% http://en.wikibooks.org/wiki/LaTeX
%
%%%%%%%%%%%%%%%%%%%%%%%%%%%%%%%%%%%%%%%%%%%%%%%%%%%%%%%%%%%%%%%%%%%%%%
%
% --------------------------------------------------------------------
% Preamble
% --------------------------------------------------------------------
\documentclass[paper=a4, fontsize=11pt,twoside]{scrartcl}	% KOMA

\usepackage[a4paper,pdftex]{geometry}	% A4paper margins
\setlength{\oddsidemargin}{5mm}			% Remove 'twosided' indentation
\setlength{\evensidemargin}{5mm}

\usepackage[english]{babel}
\usepackage[protrusion=true,expansion=true]{microtype}	
\usepackage{amsmath,amsfonts,amsthm,amssymb}
\usepackage{graphicx}
\usepackage{hyperref}
% --------------------------------------------------------------------
% Definitions (do not change this)
% --------------------------------------------------------------------
\newcommand{\HRule}[1]{\rule{\linewidth}{#1}} 	% Horizontal rule

\makeatletter							% Title
\def\printtitle{%						
    {\centering \@title\par}}
\makeatother									

\makeatletter							% Author
\def\printauthor{%					
    {\centering \large \@author}}				
\makeatother							

\hypersetup{
    colorlinks=true,
    linkcolor=blue,
    filecolor=magenta,      
    urlcolor=cyan,
}
% --------------------------------------------------------------------
% Metadata (Change this)
% --------------------------------------------------------------------
\title{	\normalsize \textsc{Informatics, UoE 2021} 	% Subtitle
		 	\\[2.0cm]								% 2cm spacing
			\HRule{0.5pt} \\						% Upper rule
			\LARGE \textbf{\uppercase{Reinforcment Learning Tutorial 1}}	% Title
			\HRule{2pt} \\ [0.5cm]		% Lower rule + 0.5cm spacing
			\normalsize \today			% Todays date
		}

\author{
		Fazl Barez\\	
        \texttt{fazlbarez93@gmail.com} \\
}


\begin{document}
% ------------------------------------------------------------------------------
% Maketitle
% ------------------------------------------------------------------------------
\thispagestyle{empty}		% Remove page numbering on this page

\printtitle					% Print the title data as defined above
  	\vfill
\printauthor				% Print the author data as defined above
\newpage
% ------------------------------------------------------------------------------
% Begin document
% ------------------------------------------------------------------------------
\setcounter{page}{1}		% Set page numbering to begin on this page
\section*{Introduction}

In this document students are encouraged to rise any questions, points from the tutorial. 
The idea is to keep a backlog of all the questions raised during and after the tutorial. Student can refer to this document for reference. 

\section*{Discussion}

\begin{itemize}
    \item {In the \href{https://en.wikipedia.org/wiki/Trolley_problem}{\textbf{Trolley Problem}} setting, who should decide what is the \textit{"correct"} decision?}
    \item {How can we use supervised learning in RL? \href{https://bair.berkeley.edu/blog/2020/10/13/supervised-rl/}{\textbf{Supervised RL?}}}
\end{itemize}
% ------------------------------------------------------------------------------
% End document
% ------------------------------------------------------------------------------
\end{document}